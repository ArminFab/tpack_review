\PassOptionsToPackage{unicode=true}{hyperref} % options for packages loaded elsewhere
\PassOptionsToPackage{hyphens}{url}
%
\documentclass[]{article}
\usepackage{lmodern}
\usepackage{amssymb,amsmath}
\usepackage{ifxetex,ifluatex}
\usepackage{fixltx2e} % provides \textsubscript
\ifnum 0\ifxetex 1\fi\ifluatex 1\fi=0 % if pdftex
  \usepackage[T1]{fontenc}
  \usepackage[utf8]{inputenc}
  \usepackage{textcomp} % provides euro and other symbols
\else % if luatex or xelatex
  \usepackage{unicode-math}
  \defaultfontfeatures{Ligatures=TeX,Scale=MatchLowercase}
\fi
% use upquote if available, for straight quotes in verbatim environments
\IfFileExists{upquote.sty}{\usepackage{upquote}}{}
% use microtype if available
\IfFileExists{microtype.sty}{%
\usepackage[]{microtype}
\UseMicrotypeSet[protrusion]{basicmath} % disable protrusion for tt fonts
}{}
\IfFileExists{parskip.sty}{%
\usepackage{parskip}
}{% else
\setlength{\parindent}{0pt}
\setlength{\parskip}{6pt plus 2pt minus 1pt}
}
\usepackage{hyperref}
\hypersetup{
            pdftitle={A Template for Systematic Reviews and Metaanalyses},
            pdfborder={0 0 0},
            breaklinks=true}
\urlstyle{same}  % don't use monospace font for urls
\usepackage[margin=1in]{geometry}
\usepackage{longtable,booktabs}
% Fix footnotes in tables (requires footnote package)
\IfFileExists{footnote.sty}{\usepackage{footnote}\makesavenoteenv{longtable}}{}
\usepackage{graphicx,grffile}
\makeatletter
\def\maxwidth{\ifdim\Gin@nat@width>\linewidth\linewidth\else\Gin@nat@width\fi}
\def\maxheight{\ifdim\Gin@nat@height>\textheight\textheight\else\Gin@nat@height\fi}
\makeatother
% Scale images if necessary, so that they will not overflow the page
% margins by default, and it is still possible to overwrite the defaults
% using explicit options in \includegraphics[width, height, ...]{}
\setkeys{Gin}{width=\maxwidth,height=\maxheight,keepaspectratio}
\setlength{\emergencystretch}{3em}  % prevent overfull lines
\providecommand{\tightlist}{%
  \setlength{\itemsep}{0pt}\setlength{\parskip}{0pt}}
\setcounter{secnumdepth}{0}
% Redefines (sub)paragraphs to behave more like sections
\ifx\paragraph\undefined\else
\let\oldparagraph\paragraph
\renewcommand{\paragraph}[1]{\oldparagraph{#1}\mbox{}}
\fi
\ifx\subparagraph\undefined\else
\let\oldsubparagraph\subparagraph
\renewcommand{\subparagraph}[1]{\oldsubparagraph{#1}\mbox{}}
\fi

% set default figure placement to htbp
\makeatletter
\def\fps@figure{htbp}
\makeatother

\usepackage{booktabs}
\usepackage{longtable}
\usepackage{array}
\usepackage{multirow}
\usepackage{wrapfig}
\usepackage{float}
\usepackage{colortbl}
\usepackage{pdflscape}
\usepackage{tabu}
\usepackage{threeparttable}
\usepackage{threeparttablex}
\usepackage[normalem]{ulem}
\usepackage{makecell}
\usepackage{xcolor}

\title{A Template for Systematic Reviews and Metaanalyses}
\author{true \and true}
\date{2020-06-03}

\begin{document}
\maketitle

{
\setcounter{tocdepth}{2}
\tableofcontents
}
\textbf{This is} a Rmd-template for protocols and reporting of
systematic reviews and meta-analyses. It synthesizes three sources of
standards:

\begin{itemize}
\tightlist
\item
  \href{https://doi.org/10.1136/bmj.i4086}{PRISMA-P}
\item
  \href{https://www.crd.york.ac.uk/prospero/}{PROSPERO}
\item
  \href{https://doi.org/10.1037/amp0000389}{MARS}
\end{itemize}

The template is \textbf{aimed at}

\begin{itemize}
\tightlist
\item
  guiding the process of planning the systemtic review/ meta-analysis
\item
  providing a form for preregistration (enter your text, export as
  standalone html, upload as preregistration)
\end{itemize}

We are aware that MARS targets aspects of reporting after the systemtic
review/ meta-analysis is completed rather than decisions and reasoning
in the planning phase as PRISMA-P and PROSPERO. MARS nevertheless
provides a good framework to determine crucial points for systemtic
reviews/ meta-analyses to be addressed as early as in the planning
phase.

\emph{Standards have been partially adapted. Click `show changes' to see
changes and reasons for change.}

show changes

\hypertarget{changes}{}
\begin{longtable}[]{@{}lll@{}}
\toprule
\begin{minipage}[b]{0.15\columnwidth}\raggedright
standard\strut
\end{minipage} & \begin{minipage}[b]{0.33\columnwidth}\raggedright
implemented change\strut
\end{minipage} & \begin{minipage}[b]{0.43\columnwidth}\raggedright
reason\strut
\end{minipage}\tabularnewline
\midrule
\endhead
\begin{minipage}[t]{0.15\columnwidth}\raggedright
MARS\strut
\end{minipage} & \begin{minipage}[t]{0.33\columnwidth}\raggedright
Left out paper section ``Abstract''\strut
\end{minipage} & \begin{minipage}[t]{0.43\columnwidth}\raggedright
An abstract is important for reporting, not however, for planning and
registering.\strut
\end{minipage}\tabularnewline
\begin{minipage}[t]{0.15\columnwidth}\raggedright
MARS\strut
\end{minipage} & \begin{minipage}[t]{0.33\columnwidth}\raggedright
Left out paper section ``Results'' and parts of ``Discussion''\strut
\end{minipage} & \begin{minipage}[t]{0.43\columnwidth}\raggedright
Specifications on how to report results is important for reporting, not
however, for planning and registering. Prospective information on how
results will be computed/ synthesized is preserved.\strut
\end{minipage}\tabularnewline
\begin{minipage}[t]{0.15\columnwidth}\raggedright
MARS\strut
\end{minipage} & \begin{minipage}[t]{0.33\columnwidth}\raggedright
Left out ``Protocol: List where the full protocol can be found''\strut
\end{minipage} & \begin{minipage}[t]{0.43\columnwidth}\raggedright
This form practically is the protocol.\strut
\end{minipage}\tabularnewline
\begin{minipage}[t]{0.15\columnwidth}\raggedright
PROSPERO\strut
\end{minipage} & \begin{minipage}[t]{0.33\columnwidth}\raggedright
Left out non-mandatory fields or integrated them with mandatory
fields.\strut
\end{minipage} & \begin{minipage}[t]{0.43\columnwidth}\raggedright
Avoiding too detailed specifications. All relevant informations will be
integrated.\strut
\end{minipage}\tabularnewline
\begin{minipage}[t]{0.15\columnwidth}\raggedright
PROSPERO\strut
\end{minipage} & \begin{minipage}[t]{0.33\columnwidth}\raggedright
Left out some options in ``Type and method of review''\strut
\end{minipage} & \begin{minipage}[t]{0.43\columnwidth}\raggedright
Options left out are purely health/ medicine related.\strut
\end{minipage}\tabularnewline
\begin{minipage}[t]{0.15\columnwidth}\raggedright
PROSPERO\strut
\end{minipage} & \begin{minipage}[t]{0.33\columnwidth}\raggedright
Left out ``Health area of the review''\strut
\end{minipage} & \begin{minipage}[t]{0.43\columnwidth}\raggedright
This field is purely health/ medicine related.\strut
\end{minipage}\tabularnewline
\bottomrule
\end{longtable}

\hypertarget{general}{%
\section{General}\label{general}}

\hypertarget{working-title}{%
\subsection{Working Title}\label{working-title}}

show sources

\hypertarget{wt}{}
\begin{table}[H]
\centering
\begin{tabular}{>{\bfseries}l|l}
\hline
\rowcolor[HTML]{ececec}  source & description\\
\hline
PRISMA-P & Identify the report as a protocol of a systematic review. If the protocol is for an update of a previous systematic review, identify as such\\
\hline
PROSPERO & Give the working title of the review, for example the one used for obtaining funding. Ideally the title should state succinctly the interventions or exposures being reviewed and the associated health or social problems. Where appropriate, the title should use the PI(E)COS structure to contain information on the Participants, Intervention (or Exposure) and Comparison groups, the Outcomes to be measured and Study designs to be included.

For reviews in languages other than English, this field should be used to enter the title in the language of the
review. This will be displayed together with the English language title.\\
\hline
MARS & Title: State the research question and type of research synthesis (e.g., narrative synthesis, meta-analysis)\\
\hline
\end{tabular}
\end{table}

\emph{Put your text here.}

\hypertarget{type-of-review}{%
\subsection{Type of Review}\label{type-of-review}}

show sources

\hypertarget{typeor}{}
\begin{table}[H]
\centering
\begin{tabular}{>{\bfseries}l|l}
\hline
\rowcolor[HTML]{ececec}  source & description\\
\hline
PRISMA-P & Not specified.\\
\hline
PROSPERO & Type and method of review: Select the type of review and the review method from the lists below. Select the health area(s) of interest for your review.

* Meta-analysis
* Narrative synthesis
* Network meta-analysis
* Review of reviews
* Synthesis of qualitative studies
* Systematic review
* Other\\
\hline
MARS & Not specified.\\
\hline
\end{tabular}
\end{table}

\emph{Put your text here.}

\hypertarget{link-to-registration}{%
\subsection{Link to Registration}\label{link-to-registration}}

show sources

\hypertarget{ltr}{}
\begin{table}[H]
\centering
\begin{tabular}{>{\bfseries}l|l}
\hline
\rowcolor[HTML]{ececec}  source & description\\
\hline
PRISMA-P & If registered, provide the name of the registry (such as PROSPERO) and registration number.\\
\hline
PROSPERO & Not specified.\\
\hline
MARS & Give the place where the synthesis is registered and its registry number, if registered\\
\hline
\end{tabular}
\end{table}

\emph{Put your text here.}

\hypertarget{anticipated-start-and-completion-date}{%
\subsection{Anticipated start and completion
date}\label{anticipated-start-and-completion-date}}

show sources

\hypertarget{aasd}{}
\begin{table}[H]
\centering
\begin{tabular}{>{\bfseries}l|l}
\hline
\rowcolor[HTML]{ececec}  source & description\\
\hline
PRISMA-P & Not specified.\\
\hline
PROSPERO & Give the date when the systematic review commenced, or is expected to commence. Give the date by which the review is expected to be completed.\\
\hline
MARS & Not specified.\\
\hline
\end{tabular}
\end{table}

\emph{Put your text here.}

\hypertarget{stage-of-review}{%
\subsection{Stage of Review}\label{stage-of-review}}

show sources

\hypertarget{sor}{}
\begin{table}[H]
\centering
\begin{tabular}{>{\bfseries}l|l}
\hline
\rowcolor[HTML]{ececec}  source & description\\
\hline
PRISMA-P & Not specified.\\
\hline
PROSPERO & Indicate the stage of progress of the review by ticking the relevant Started and Completed boxes. Additional
information may be added in the free text box provided.
Please note: Reviews that have progressed beyond the point of completing data extraction at the time of
initial registration are not eligible for inclusion in PROSPERO. Should evidence of incorrect status and/or
completion date being supplied at the time of submission come to light, the content of the PROSPERO
record will be removed leaving only the title and named contact details and a statement that inaccuracies in
the stage of the review date had been identified.
This field should be updated when any amendments are made to a published record and on completion and
publication of the review. If this field was pre-populated from the initial screening questions then you are not
able to edit it until the record is published.

* The review has not yet started: [yes/no]

| Review stage | Started | Completed |
|:--------------------------------------------| :----:| :----:|
| Preliminary searches | Yes/No | Yes/No
| Piloting of the study selection process | Yes/No | Yes/No
| Formal screening of search results against eligibility criteria | Yes/No | Yes/No
| Data extraction | Yes/No | Yes/No
| Risk of bias (quality) assessment | Yes/No | Yes/No
| Data analysis | Yes/No | Yes/No

Provide any other relevant information about the stage of the review here (e.g. Funded proposal, protocol not
yet finalised).\\
\hline
MARS & Not specified.\\
\hline
\end{tabular}
\end{table}

\emph{Put your text here.}

\hypertarget{names-affiliations-contact}{%
\subsection{Names, Affiliations,
Contact}\label{names-affiliations-contact}}

show sources

\hypertarget{nac}{}
\begin{table}[H]
\centering
\begin{tabular}{>{\bfseries}l|l}
\hline
\rowcolor[HTML]{ececec}  source & description\\
\hline
PRISMA-P & * Provide name, institutional affiliation, e-mail address of all protocol authors; provide physical mailing address of corresponding author.
* Describe contributions of protocol authors and identify the guarantor of the review.\\
\hline
PROSPERO & * Named Contact: The named contact acts as the guarantor for the accuracy of the information presented in the register record.
* Named contact email: Give the electronic mail address of the named contact.
* Organisational affiliation of the review: Full title of the organisational affiliations for this review and website address if available. This field may be completed as 'None' if the review is not affiliated to any organisation.
* Review team members and their organisational affiliations: Give the personal details and the organisational affiliations of each member of the review team. Affiliation refers to groups or organisations to which review team members belong.\\
\hline
MARS & Not specified.\\
\hline
\end{tabular}
\end{table}

\emph{Put your text here.}

\hypertarget{collaborators}{%
\subsection{Collaborators}\label{collaborators}}

show sources

\hypertarget{colla}{}
\begin{table}[H]
\centering
\begin{tabular}{>{\bfseries}l|l}
\hline
\rowcolor[HTML]{ececec}  source & description\\
\hline
PRISMA-P & Not specified.\\
\hline
PROSPERO & Collaborators (name \& affilitation) of individuals working on the review, but are not review team member.\\
\hline
MARS & Not specified.\\
\hline
\end{tabular}
\end{table}

\emph{Put your text here.}

\hypertarget{amendments-to-previous-versions}{%
\subsection{Amendments to previous
versions}\label{amendments-to-previous-versions}}

show sources

\hypertarget{atpv}{}
\begin{table}[H]
\centering
\begin{tabular}{>{\bfseries}l|l}
\hline
\rowcolor[HTML]{ececec}  source & description\\
\hline
PRISMA-P & If the protocol represents an amendment of a previously completed or published protocol, identify as such and list changes; otherwise, state plan for documenting important protocol amendments.\\
\hline
PROSPERO & Not specified.\\
\hline
MARS & Not specified.\\
\hline
\end{tabular}
\end{table}

\emph{Put your text here.}

\hypertarget{funding-sources-sponsors-and-their-roles}{%
\subsection{Funding sources, sponsors and their
roles}\label{funding-sources-sponsors-and-their-roles}}

show sources

\hypertarget{fsign}{}
\begin{table}[H]
\centering
\begin{tabular}{>{\bfseries}l|l}
\hline
\rowcolor[HTML]{ececec}  source & description\\
\hline
PRISMA-P & * Indicate sources of financial or other support for the review
* Provide name for the review funder and/or sponsor
* Describe roles of funder(s), sponsor(s), and/or institution(s), if any, in developing the protocol\\
\hline
PROSPERO & Funding sources/sponsors: Give details of the individuals, organizations, groups or other legal entities who take responsibility for initiating, managing, sponsoring and/or financing the review. Include any unique identification numbers assigned to the review by the individuals or bodies listed. Grant numbers.\\
\hline
MARS & * List all sources of monetary and in-kind funding support
* State the role of funders in conducting the synthesis and deciding to publish the results, if any\\
\hline
\end{tabular}
\end{table}

\emph{Put your text here.}

\hypertarget{conflict-of-interest}{%
\subsection{Conflict of Interest}\label{conflict-of-interest}}

show sources

\hypertarget{coi}{}
\begin{table}[H]
\centering
\begin{tabular}{>{\bfseries}l|l}
\hline
\rowcolor[HTML]{ececec}  source & description\\
\hline
PRISMA-P & Not specified.\\
\hline
PROSPERO & List any conditions that could lead to actual or perceived undue influence on judgements concerning the main topic investigated in the review.\\
\hline
MARS & Describe possible conflicts of interest, including financial and other nonfinancial interests.\\
\hline
\end{tabular}
\end{table}

\emph{Put your text here.}

\hypertarget{introduction}{%
\section{Introduction}\label{introduction}}

\hypertarget{rationale}{%
\subsection{Rationale}\label{rationale}}

show sources

\hypertarget{ram}{}
\begin{table}[H]
\centering
\begin{tabular}{>{\bfseries}l|l}
\hline
\rowcolor[HTML]{ececec}  source & description\\
\hline
PRISMA-P & Describe the rationale for the review in the context of what is already known.\\
\hline
PROSPERO & Not specified.\\
\hline
MARS & Problem: State the question or relation(s) under investigation, including

* Historical background, including previous syntheses and meta-analyses related to the topic 
* Theoretical, policy, and/or practical issues related to the question or relation(s) of interest
* Populations and settings to which the question or relation(s) is relevant
* Rationale for
   (a) choice of study designs, 
   (b) the selection and coding of outcomes, 
   (c) the selection and coding potential moderators or mediators of results 
* Psychometric characteristics of outcome measures and other variables\\
\hline
\end{tabular}
\end{table}

\emph{Put your text here.}

\hypertarget{research-questions}{%
\subsection{Research Questions}\label{research-questions}}

show sources

\hypertarget{rq}{}
\begin{table}[H]
\centering
\begin{tabular}{>{\bfseries}l|l}
\hline
\rowcolor[HTML]{ececec}  source & description\\
\hline
PRISMA-P & Provide an explicit statement of the question(s) the review will address with reference to participants, interventions, comparators, and outcomes (PICO)\\
\hline
PROSPERO & State the question(s) to be addressed by the review, clearly and precisely. Review questions may be specific or broad. It may be appropriate to break very broad questions down into a series of related more specific questions. Questions may be framed or refined using PI(E)COS where relevant.\\
\hline
MARS & Objectives: State the hypotheses examined, indicating which were prespecified, including

* Question in terms of relevant participant characteristics (including animal populations), independent variables (experimental manipulations, treatments, or interventions), ruling out of possible confounding variables, dependent variables (outcomes, criterion), and other features of study designs
* Method(s) of synthesis and if meta-analysis was used, the specific methods used to integrate studies (e.g., effect-size metric, averaging method, the model used in homogeneity analysis)\\
\hline
\end{tabular}
\end{table}

\emph{Put your text here.}

\hypertarget{methods}{%
\section{Methods}\label{methods}}

\hypertarget{eligibility-inclusion-and-exclusion-criteria}{%
\subsection{Eligibility: Inclusion and Exclusion
Criteria}\label{eligibility-inclusion-and-exclusion-criteria}}

show sources

\hypertarget{eiaec}{}
\begin{table}[H]
\centering
\begin{tabular}{>{\bfseries}l|l}
\hline
\rowcolor[HTML]{ececec}  source & description\\
\hline
PRISMA-P & Specify the study characteristics (such as PICO, study design, setting, time frame) and report characteristics (such as years considered, language, publication status) to be used as criteria for eligibility for the review\\
\hline
PROSPERO & Give details of the types of study (study designs) eligible for inclusion in the review. If there are no restrictions on the types of study design eligible for inclusion, or certain study types are excluded, this should be stated. The preferred format includes details of both inclusion and exclusion criteria.\\
\hline
MARS & Describe the criteria for selecting studies, including

* Independent variables (e.g., experimental manipulations, types of treatments or interventions or predictor variables)
* Dependent variable (e.g., outcomes, in syntheses of clinical research including both potential benefits and potential adverse effects)
* Eligible study designs (e.g., methods of sampling or treatment assignment)
* Handling of multiple reports about the same study or sample, describing which are primary and handling of multiple measures using the same participants
* Restrictions on study inclusion (e.g., by study age, language, location, or report type)
* Changes to the prespecified inclusion and exclusion criteria, and when these changes were made
* Handling of reports that did not contain sufficient information to judge eligibility (e.g., lacking information about study design) and reports that did not include sufficient information for analysis (e.g., did not report numerical data about those outcomes)\\
\hline
\end{tabular}
\end{table}

\emph{Put your text here.}

\hypertarget{sources-of-search-list-and-rationale}{%
\subsection{Sources of Search: List and
Rationale}\label{sources-of-search-list-and-rationale}}

show sources

\hypertarget{soslar}{}
\begin{table}[H]
\centering
\begin{tabular}{>{\bfseries}l|l}
\hline
\rowcolor[HTML]{ececec}  source & description\\
\hline
PRISMA-P & Not specified.\\
\hline
PROSPERO & Searches: State the sources that will be searched. Give the search dates, and any restrictions (e.g. language or
publication period). Do NOT enter the full search strategy (it may be provided as a link or attachment.)\\
\hline
MARS & Describe all information sources:

* Databases searched (e.g., PsycINFO, ClinicalTrials.gov), including dates of coverage (i.e., earliest and latest records included in the search), and software and search platforms used
* Names of specific journals that were searched and the volumes checked
* Explanation of rationale for choosing reference lists if examined (e.g., other relevant articles, previous research
syntheses)
* Documents for which forward (citation) searches were conducted, stating why these documents were chosen
* Number of researchers contacted if study authors or individual researchers were contacted to find studies or to obtain more information about included studies, as well as criteria for making contact (e.g., previous relevant publications), and response rate
* Dates of contact if other direct contact searches were conducted such as contacting corporate sponsors or mailings to distribution lists
* Search strategies in addition to those above and the results of these searches\\
\hline
\end{tabular}
\end{table}

\emph{Put your text here.}

\hypertarget{search-strategy}{%
\subsection{Search Strategy}\label{search-strategy}}

show sources

\hypertarget{searchs}{}
\begin{table}[H]
\centering
\begin{tabular}{>{\bfseries}l|l}
\hline
\rowcolor[HTML]{ececec}  source & description\\
\hline
PRISMA-P & Present draft of search strategy to be used for at least one electronic database, including planned limits, such that it could be repeated.\\
\hline
PROSPERO & URL to search strategy: Give a link to a published pdf/word document detailing either the search strategy or an example of a search strategy for a specific database if available (including the keywords that will be used in the search strategies), or upload your search strategy. Do NOT provide links to your search results. Alternatively, upload your search strategy to CRD in pdf format. Please note that by doing so you are consenting to the file being made publicly accessible.\\
\hline
MARS & Describe all information sources: Search strategies of electronic searches, such that they could be repeated (e.g., include the search terms used, Boolean connectors, fields searched, explosion of terms).\\
\hline
\end{tabular}
\end{table}

\emph{Put your text here.}

\hypertarget{data-management-tools-used}{%
\subsection{Data Management Tools
Used}\label{data-management-tools-used}}

show sources

\hypertarget{dmtu}{}
\begin{table}[H]
\centering
\begin{tabular}{>{\bfseries}l|l}
\hline
\rowcolor[HTML]{ececec}  source & description\\
\hline
PRISMA-P & Describe the mechanism(s) that will be used to manage records and data throughout the review.\\
\hline
PROSPERO & Not specified.\\
\hline
MARS & Not specified.\\
\hline
\end{tabular}
\end{table}

\emph{Put your text here.}

\hypertarget{data-extraction-selection-of-studies}{%
\subsection{Data Extraction (Selection of
Studies)}\label{data-extraction-selection-of-studies}}

show sources

\hypertarget{desos}{}
\begin{table}[H]
\centering
\begin{tabular}{>{\bfseries}l|l}
\hline
\rowcolor[HTML]{ececec}  source & description\\
\hline
PRISMA-P & State the process that will be used for selecting studies (such as two independent reviewers) through each phase of the review (that is, screening, eligibility and inclusion in meta-analysis).\\
\hline
PROSPERO & Data extraction (selection and coding): Describe how studies will be selected for inclusion. State what data will be extracted or obtained. State how this will be done and recorded.\\
\hline
MARS & Describe the process for deciding which studies would be included in the syntheses and/or included in the meta-analysis, including

* Document elements (e.g., title, abstract, full text) used to make decisions about inclusion or exclusion from the synthesis at each step of the screening process 
* Qualifications (e.g., training, educational or professional status) of those who conducted each step in the study selection process, stating whether each step was conducted by a single person or in duplicate as well as an explanation of how reliability was assessed if one screener was used and how disagreements were resolved if multiple were used.\\
\hline
\end{tabular}
\end{table}

\emph{Put your text here.}

\hypertarget{method-of-extracting-data-information-from-reports}{%
\subsection{Method of Extracting Data \& Information (from
Reports)}\label{method-of-extracting-data-information-from-reports}}

show sources

\hypertarget{edfr}{}
\begin{table}[H]
\centering
\begin{tabular}{>{\bfseries}l|l}
\hline
\rowcolor[HTML]{ececec}  source & description\\
\hline
PRISMA-P & Describe planned method of extracting data from reports (such as piloting forms, done independently, in duplicate), any processes for obtaining and confirming data from investigators.\\
\hline
PROSPERO & Not specified.\\
\hline
MARS & Describe methods of extracting data from reports, including 

* Variables for which data were sought and the variable categories 
* Qualifications of those who conducted each step in the data extraction process, stating whether each step was conducted by a single person or in duplicate and an explanation of how reliability was assessed if one screener was used and how disagreements were resolved if multiple screeners were used as well as whether data coding forms, instructions for completion, and the data (including metadata) are available, stating where they can be found (e.g., public registry, supplemental materials)\\
\hline
\end{tabular}
\end{table}

\emph{Put your text here.}

\hypertarget{list-and-description-of-data-and-information-extracted}{%
\subsection{List and Description of Data and Information
Extracted}\label{list-and-description-of-data-and-information-extracted}}

show sources

\hypertarget{ladodaie}{}
\begin{table}[H]
\centering
\begin{tabular}{>{\bfseries}l|l}
\hline
\rowcolor[HTML]{ececec}  source & description\\
\hline
PRISMA-P & * List and define all variables for which data will be sought (such as PICO items, funding sources), any pre-planned data assumptions and simplifications
* List and define all outcomes for which data will be sought, including prioritization of main and additional outcomes, with rationale\\
\hline
PROSPERO & * Condition or domain being studied: Give a short description of the disease, condition or healthcare domain being studied. This could include health and wellbeing outcomes.
* Participants/population: Give summary criteria for the participants or populations being studied by the review. The preferred format includes details of both inclusion and exclusion criteria.
* Intervention(s), exposure(s): Give full and clear descriptions or definitions of the nature of the interventions or the exposures to be reviewed.
* Comparator(s)/control: Where relevant, give details of the alternatives against which the main subject/topic of the review will be compared (e.g. another intervention or a non-exposed control group). The preferred format includes details of both inclusion and exclusion criteria.
* Main and additional outcome(s): Give the pre-specified main (most important) outcomes of the review, including details of how the outcome is defined and measured and when these measurement are made, if these are part of the review inclusion criteria.
* Measures of effect: Please specify the effect measure(s) for you main outcome(s) e.g. relative risks, odds ratios, risk difference,
and/or 'number needed to treat.\\
\hline
MARS & Not specified.\\
\hline
\end{tabular}
\end{table}

\emph{Put your text here.}

\hypertarget{effect-size-transformation-from-individual-studies}{%
\subsection{Effect size transformation from individual
studies}\label{effect-size-transformation-from-individual-studies}}

show sources

\hypertarget{estfis}{}
\begin{table}[H]
\centering
\begin{tabular}{>{\bfseries}l|l}
\hline
\rowcolor[HTML]{ececec}  source & description\\
\hline
PRISMA-P & Not specified.\\
\hline
PROSPERO & Not specified.\\
\hline
MARS & Describe the statistical methods for calculating effect sizes, including the metric(s) used (e.g., correlation coefficients,
differences in means, risk ratios) and formula(s) used to calculate effect sizes.\\
\hline
\end{tabular}
\end{table}

\emph{Put your text here.}

\hypertarget{risk-of-bias-in-individual-studies}{%
\subsection{Risk of Bias in Individual
Studies}\label{risk-of-bias-in-individual-studies}}

show sources

\hypertarget{robiis}{}
\begin{table}[H]
\centering
\begin{tabular}{>{\bfseries}l|l}
\hline
\rowcolor[HTML]{ececec}  source & description\\
\hline
PRISMA-P & Describe anticipated methods for assessing risk of bias of individual studies, including whether this will be done at the outcome or study level, or both; state how this information will be used in data synthesis\\
\hline
PROSPERO & Risk of bias (quality) assessment: Describe the method of assessing risk of bias or quality assessment. State which characteristics of the studies will be assessed and any formal risk of bias tools that will be used.\\
\hline
MARS & Describe any methods used to assess risk to internal validity in individual study results, including

* Risks assessed and criteria for concluding risk exists or does not exist
* Methods for including risk to internal validity in the decisions to synthesize of the data and the interpretation of results\\
\hline
\end{tabular}
\end{table}

\emph{Put your text here.}

\hypertarget{results}{%
\section{Results}\label{results}}

\hypertarget{strategy-for-data-sythesis}{%
\subsection{Strategy for Data
Sythesis}\label{strategy-for-data-sythesis}}

show sources

\hypertarget{sfds}{}
\begin{table}[H]
\centering
\begin{tabular}{>{\bfseries}l|l}
\hline
\rowcolor[HTML]{ececec}  source & description\\
\hline
PRISMA-P & * Describe criteria under which study data will be quantitatively synthesised.
* If data are appropriate for quantitative synthesis, describe planned summary measures, methods of handling data and methods of combining data from studies, including any planned exploration of consistency (such as I2, Kendall’s τ)
* If quantitative synthesis is not appropriate, describe the type of summary planned\\
\hline
PROSPERO & Strategy for data synthesis: Provide details of the planned synthesis including a rationale for the methods selected. This must not be generic text but should be specific to your review and describe how the proposed analysis will be applied to your data.\\
\hline
MARS & Describe narrative and statistical methods used to compare studies. If meta-analysis was conducted, describe the methods used to combine effects across studies and the model used to estimate the heterogeneity of the effects sizes (e.g., a fixed-effect, random-effects model robust variance estimation), including

* Rationale for the method of synthesis
* Methods for weighting study results
* Methods to estimate imprecision (e.g., confidence or credibility intervals) both within and between studies
* Description of all transformations or corrections (e.g., to account for small samples or unequal group numbers) and adjustments (e.g., for clustering, missing data, measurement artifacts, or construct-level relationships) made to the data and justification for these
* Additional analyses (e.g., subgroup analyses, meta-regression), including whether each analysis was prespecified or post hoc
* Selection of prior distributions and assessment of model fit if Bayesian analyses were conducted
* Name and version number of computer programs used for the analysis
* Statistical code and where it can be found (e.g., a supplement)\\
\hline
\end{tabular}
\end{table}

\emph{Put your text here.}

\hypertarget{moderators-subgroups}{%
\subsection{Moderators/ Subgroups}\label{moderators-subgroups}}

show sources

\hypertarget{modarat}{}
\begin{table}[H]
\centering
\begin{tabular}{>{\bfseries}l|l}
\hline
\rowcolor[HTML]{ececec}  source & description\\
\hline
PRISMA-P & Describe any proposed additional analyses (such as sensitivity or subgroup analyses, meta-regression)\\
\hline
PROSPERO & Analysis of subgroups or subsets: State any planned investigation of ‘subgroups’. Be clear and specific about which type of study or participant will be included in each group or covariate investigated. State the planned analytic approach.\\
\hline
MARS & Not specified.\\
\hline
\end{tabular}
\end{table}

\emph{Put your text here.}

\hypertarget{assessment-of-publication-bias}{%
\subsection{Assessment of Publication
Bias}\label{assessment-of-publication-bias}}

show sources

\hypertarget{aopb}{}
\begin{table}[H]
\centering
\begin{tabular}{>{\bfseries}l|l}
\hline
\rowcolor[HTML]{ececec}  source & description\\
\hline
PRISMA-P & Specify any planned assessment of meta-bias(es) (such as publication bias across studies, selective reporting within studies)\\
\hline
PROSPERO & Not specified.\\
\hline
MARS & Describe risk of bias across studies, including

* Statement about whether
   (a) unpublished studies and unreported data, or 
   (b) only published data were included in the synthesis and the rationale if only published data were used
* Assessments of the impact of publication bias (e.g., modeling of data censoring, trim-and-fill analysis)
* Results of any statistical analyses looking for selective reporting of results within studies\\
\hline
\end{tabular}
\end{table}

\emph{Put your text here.}

\hypertarget{discussion}{%
\section{Discussion}\label{discussion}}

\hypertarget{strength-of-evidence}{%
\subsection{Strength of Evidence}\label{strength-of-evidence}}

show sources

\hypertarget{stroe}{}
\begin{table}[H]
\centering
\begin{tabular}{>{\bfseries}l|l}
\hline
\rowcolor[HTML]{ececec}  source & description\\
\hline
PRISMA-P & Describe how the strength of the body of evidence will be assessed (such as GRADE).\\
\hline
PROSPERO & Not specified.\\
\hline
MARS & Describe the generalizability (external validity) of conclusions, including • Implications for related populations, intervention variations, dependent (outcome) variables.\\
\hline
\end{tabular}
\end{table}

\emph{Put your text here.}

\end{document}
