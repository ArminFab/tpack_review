\PassOptionsToPackage{unicode=true}{hyperref} % options for packages loaded elsewhere
\PassOptionsToPackage{hyphens}{url}
%
\documentclass[
]{article}
\usepackage{lmodern}
\usepackage{amssymb,amsmath}
\usepackage{ifxetex,ifluatex}
\ifnum 0\ifxetex 1\fi\ifluatex 1\fi=0 % if pdftex
  \usepackage[T1]{fontenc}
  \usepackage[utf8]{inputenc}
  \usepackage{textcomp} % provides euro and other symbols
\else % if luatex or xelatex
  \usepackage{unicode-math}
  \defaultfontfeatures{Scale=MatchLowercase}
  \defaultfontfeatures[\rmfamily]{Ligatures=TeX,Scale=1}
\fi
% use upquote if available, for straight quotes in verbatim environments
\IfFileExists{upquote.sty}{\usepackage{upquote}}{}
\IfFileExists{microtype.sty}{% use microtype if available
  \usepackage[]{microtype}
  \UseMicrotypeSet[protrusion]{basicmath} % disable protrusion for tt fonts
}{}
\makeatletter
\@ifundefined{KOMAClassName}{% if non-KOMA class
  \IfFileExists{parskip.sty}{%
    \usepackage{parskip}
  }{% else
    \setlength{\parindent}{0pt}
    \setlength{\parskip}{6pt plus 2pt minus 1pt}}
}{% if KOMA class
  \KOMAoptions{parskip=half}}
\makeatother
\usepackage{xcolor}
\IfFileExists{xurl.sty}{\usepackage{xurl}}{} % add URL line breaks if available
\IfFileExists{bookmark.sty}{\usepackage{bookmark}}{\usepackage{hyperref}}
\hypersetup{
  pdftitle={Conceptualizations of the TPACK Framework},
  pdfborder={0 0 0},
  breaklinks=true}
\urlstyle{same}  % don't use monospace font for urls
\usepackage[margin=1in]{geometry}
\usepackage{longtable,booktabs}
% Allow footnotes in longtable head/foot
\IfFileExists{footnotehyper.sty}{\usepackage{footnotehyper}}{\usepackage{footnote}}
\makesavenoteenv{longtable}
\usepackage{graphicx,grffile}
\makeatletter
\def\maxwidth{\ifdim\Gin@nat@width>\linewidth\linewidth\else\Gin@nat@width\fi}
\def\maxheight{\ifdim\Gin@nat@height>\textheight\textheight\else\Gin@nat@height\fi}
\makeatother
% Scale images if necessary, so that they will not overflow the page
% margins by default, and it is still possible to overwrite the defaults
% using explicit options in \includegraphics[width, height, ...]{}
\setkeys{Gin}{width=\maxwidth,height=\maxheight,keepaspectratio}
\setlength{\emergencystretch}{3em}  % prevent overfull lines
\providecommand{\tightlist}{%
  \setlength{\itemsep}{0pt}\setlength{\parskip}{0pt}}
\setcounter{secnumdepth}{-2}
% Redefines (sub)paragraphs to behave more like sections
\ifx\paragraph\undefined\else
  \let\oldparagraph\paragraph
  \renewcommand{\paragraph}[1]{\oldparagraph{#1}\mbox{}}
\fi
\ifx\subparagraph\undefined\else
  \let\oldsubparagraph\subparagraph
  \renewcommand{\subparagraph}[1]{\oldsubparagraph{#1}\mbox{}}
\fi

% set default figure placement to htbp
\makeatletter
\def\fps@figure{htbp}
\makeatother

\usepackage{booktabs}
\usepackage{longtable}
\usepackage{array}
\usepackage{multirow}
\usepackage{wrapfig}
\usepackage{float}
\usepackage{colortbl}
\usepackage{pdflscape}
\usepackage{tabu}
\usepackage{threeparttable}
\usepackage{threeparttablex}
\usepackage[normalem]{ulem}
\usepackage{makecell}

\title{Conceptualizations of the TPACK Framework}
\author{true \and true \and true}
\date{2020-10-26}

\begin{document}
\maketitle

{
\setcounter{tocdepth}{2}
\tableofcontents
}
\textbf{This is} a Rmd-template for protocols and reporting of
systematic reviews and meta-analyses. It synthesizes three sources of
standards:

\begin{itemize}
\tightlist
\item
  \href{https://doi.org/10.1136/bmj.i4086}{PRISMA-P}
\item
  \href{https://www.crd.york.ac.uk/prospero/}{PROSPERO}
\item
  \href{https://doi.org/10.1037/amp0000389}{MARS}
\end{itemize}

The template is \textbf{aimed at}

\begin{itemize}
\tightlist
\item
  guiding the process of planning the systemtic review/ meta-analysis
\item
  providing a form for preregistration (enter your text, export as
  standalone html, upload as preregistration)
\end{itemize}

We are aware that MARS targets aspects of reporting after the systemtic
review/ meta-analysis is completed rather than decisions and reasoning
in the planning phase as PRISMA-P and PROSPERO. MARS nevertheless
provides a good framework to determine crucial points for systemtic
reviews/ meta-analyses to be addressed as early as in the planning
phase.

\emph{Standards have been partially adapted. Click `show changes' to see
changes and reasons for change.}

show changes

\hypertarget{changes}{}
\begin{collapse}

\begin{longtable}[]{@{}lll@{}}
\toprule
\begin{minipage}[b]{0.15\columnwidth}\raggedright
standard\strut
\end{minipage} & \begin{minipage}[b]{0.33\columnwidth}\raggedright
implemented change\strut
\end{minipage} & \begin{minipage}[b]{0.43\columnwidth}\raggedright
reason\strut
\end{minipage}\tabularnewline
\midrule
\endhead
\begin{minipage}[t]{0.15\columnwidth}\raggedright
MARS\strut
\end{minipage} & \begin{minipage}[t]{0.33\columnwidth}\raggedright
Left out paper section ``Abstract''\strut
\end{minipage} & \begin{minipage}[t]{0.43\columnwidth}\raggedright
An abstract is important for reporting, not however, for planning and
registering.\strut
\end{minipage}\tabularnewline
\begin{minipage}[t]{0.15\columnwidth}\raggedright
MARS\strut
\end{minipage} & \begin{minipage}[t]{0.33\columnwidth}\raggedright
Left out paper section ``Results'' and parts of ``Discussion''\strut
\end{minipage} & \begin{minipage}[t]{0.43\columnwidth}\raggedright
Specifications on how to report results is important for reporting, not
however, for planning and registering. Prospective information on how
results will be computed/ synthesized is preserved.\strut
\end{minipage}\tabularnewline
\begin{minipage}[t]{0.15\columnwidth}\raggedright
MARS\strut
\end{minipage} & \begin{minipage}[t]{0.33\columnwidth}\raggedright
Left out ``Protocol: List where the full protocol can be found''\strut
\end{minipage} & \begin{minipage}[t]{0.43\columnwidth}\raggedright
This form practically is the protocol.\strut
\end{minipage}\tabularnewline
\begin{minipage}[t]{0.15\columnwidth}\raggedright
PROSPERO\strut
\end{minipage} & \begin{minipage}[t]{0.33\columnwidth}\raggedright
Left out non-mandatory fields or integrated them with mandatory
fields.\strut
\end{minipage} & \begin{minipage}[t]{0.43\columnwidth}\raggedright
Avoiding too detailed specifications. All relevant informations will be
integrated.\strut
\end{minipage}\tabularnewline
\begin{minipage}[t]{0.15\columnwidth}\raggedright
PROSPERO\strut
\end{minipage} & \begin{minipage}[t]{0.33\columnwidth}\raggedright
Left out some options in ``Type and method of review''\strut
\end{minipage} & \begin{minipage}[t]{0.43\columnwidth}\raggedright
Options left out are purely health/ medicine related.\strut
\end{minipage}\tabularnewline
\begin{minipage}[t]{0.15\columnwidth}\raggedright
PROSPERO\strut
\end{minipage} & \begin{minipage}[t]{0.33\columnwidth}\raggedright
Left out ``Health area of the review''\strut
\end{minipage} & \begin{minipage}[t]{0.43\columnwidth}\raggedright
This field is purely health/ medicine related.\strut
\end{minipage}\tabularnewline
\bottomrule
\end{longtable}

\end{collapse}

\hypertarget{general}{%
\section{General}\label{general}}

\hypertarget{working-title}{%
\subsection{Working Title}\label{working-title}}

show sources

\hypertarget{wt}{}
\begin{collapse}

\begin{table}[H]
\centering
\begin{tabular}{>{}l|l}
\hline
\cellcolor[HTML]{ececec}{source} & \cellcolor[HTML]{ececec}{description}\\
\hline
\textbf{PRISMA-P} & Identify the report as a protocol of a systematic review. If the protocol is for an update of a previous systematic review, identify as such\\
\hline
PROSPERO & Give the working title of the review, for example the one used for obtaining funding. Ideally the title should state succinctly the interventions or exposures being reviewed and the associated health or social problems. Where appropriate, the title should use the PI(E)COS structure to contain information on the Participants, Intervention (or Exposure) and Comparison groups, the Outcomes to be measured and Study designs to be included.

For reviews in languages other than English, this field should be used to enter the title in the language of the
\textbf{review. This will be displayed together with the English language title.}\\
\hline
\textbf{MARS} & Title: State the research question and type of research synthesis (e.g., narrative synthesis, meta-analysis)\\
\hline
\end{tabular}
\end{table}

\end{collapse}

Cleaning up the Mess : A Systematic Review on the Diverse
Conceptualizations of the Technological, Pedagogical and Content
Knowledge (TPACK) Framework

\hypertarget{type-of-review}{%
\subsection{Type of Review}\label{type-of-review}}

show sources

\hypertarget{typeor}{}
\begin{collapse}

\begin{table}[H]
\centering
\begin{tabular}{>{}l|l}
\hline
\cellcolor[HTML]{ececec}{source} & \cellcolor[HTML]{ececec}{description}\\
\hline
\textbf{PRISMA-P} & Not specified.\\
\hline
PROSPERO & Type and method of review: Select the type of review and the review method from the lists below. Select the health area(s) of interest for your review.

* Meta-analysis
* Narrative synthesis
* Network meta-analysis
* Review of reviews
* Synthesis of qualitative studies
* Systematic review
\textbf{* Other}\\
\hline
\textbf{MARS} & Not specified.\\
\hline
\end{tabular}
\end{table}

\end{collapse}

Systematic Review

\hypertarget{link-to-registration}{%
\subsection{Link to Registration}\label{link-to-registration}}

show sources

\hypertarget{ltr}{}
\begin{collapse}

\begin{table}[H]
\centering
\begin{tabular}{>{}l|l}
\hline
\cellcolor[HTML]{ececec}{source} & \cellcolor[HTML]{ececec}{description}\\
\hline
\textbf{PRISMA-P} & If registered, provide the name of the registry (such as PROSPERO) and registration number.\\
\hline
\textbf{PROSPERO} & Not specified.\\
\hline
\textbf{MARS} & Give the place where the synthesis is registered and its registry number, if registered\\
\hline
\end{tabular}
\end{table}

\end{collapse}

This form is used as registration

\hypertarget{anticipated-start-and-completion-date}{%
\subsection{Anticipated start and completion
date}\label{anticipated-start-and-completion-date}}

show sources

\hypertarget{aasd}{}
\begin{collapse}

\begin{table}[H]
\centering
\begin{tabular}{>{}l|l}
\hline
\cellcolor[HTML]{ececec}{source} & \cellcolor[HTML]{ececec}{description}\\
\hline
\textbf{PRISMA-P} & Not specified.\\
\hline
\textbf{PROSPERO} & Give the date when the systematic review commenced, or is expected to commence. Give the date by which the review is expected to be completed.\\
\hline
\textbf{MARS} & Not specified.\\
\hline
\end{tabular}
\end{table}

\end{collapse}

\textbf{Start}: October 2020

\textbf{Anticipated Completion Date}: July 2021

\hypertarget{stage-of-review}{%
\subsection{Stage of Review}\label{stage-of-review}}

show sources

\hypertarget{sor}{}
\begin{collapse}

\begin{table}[H]
\centering
\begin{tabular}{>{}l|l}
\hline
\cellcolor[HTML]{ececec}{source} & \cellcolor[HTML]{ececec}{description}\\
\hline
\textbf{PRISMA-P} & Not specified.\\
\hline
PROSPERO & Indicate the stage of progress of the review by ticking the relevant Started and Completed boxes. Additional
information may be added in the free text box provided.
Please note: Reviews that have progressed beyond the point of completing data extraction at the time of
initial registration are not eligible for inclusion in PROSPERO. Should evidence of incorrect status and/or
completion date being supplied at the time of submission come to light, the content of the PROSPERO
record will be removed leaving only the title and named contact details and a statement that inaccuracies in
the stage of the review date had been identified.
This field should be updated when any amendments are made to a published record and on completion and
publication of the review. If this field was pre-populated from the initial screening questions then you are not
able to edit it until the record is published.

* The review has not yet started: [yes/no]

| Review stage | Started | Completed |
|:--------------------------------------------| :----:| :----:|
| Preliminary searches | Yes/No | Yes/No
| Piloting of the study selection process | Yes/No | Yes/No
| Formal screening of search results against eligibility criteria | Yes/No | Yes/No
| Data extraction | Yes/No | Yes/No
| Risk of bias (quality) assessment | Yes/No | Yes/No
| Data analysis | Yes/No | Yes/No

Provide any other relevant information about the stage of the review here (e.g. Funded proposal, protocol not
\textbf{yet finalised).}\\
\hline
\textbf{MARS} & Not specified.\\
\hline
\end{tabular}
\end{table}

\end{collapse}

The review has not yet started {[}yes/no{]}: no

\begin{longtable}[]{@{}lcc@{}}
\toprule
Review stage & Started & Completed\tabularnewline
\midrule
\endhead
Preliminary searches & Yes & Yes\tabularnewline
Piloting of the study selection process & No & No\tabularnewline
Formal screening of search results against eligibility criteria & No &
No\tabularnewline
Data extraction & No & No\tabularnewline
Risk of bias (quality) assessment & No & No\tabularnewline
Data analysis & No & No\tabularnewline
\bottomrule
\end{longtable}

\hypertarget{names-affiliations-contact}{%
\subsection{Names, Affiliations,
Contact}\label{names-affiliations-contact}}

show sources

\hypertarget{nac}{}
\begin{collapse}

\begin{table}[H]
\centering
\begin{tabular}{>{}l|l}
\hline
\cellcolor[HTML]{ececec}{source} & \cellcolor[HTML]{ececec}{description}\\
\hline
PRISMA-P & * Provide name, institutional affiliation, e-mail address of all protocol authors; provide physical mailing address of corresponding author.
\textbf{* Describe contributions of protocol authors and identify the guarantor of the review.}\\
\hline
PROSPERO & * Named Contact: The named contact acts as the guarantor for the accuracy of the information presented in the register record.
* Named contact email: Give the electronic mail address of the named contact.
* Organisational affiliation of the review: Full title of the organisational affiliations for this review and website address if available. This field may be completed as 'None' if the review is not affiliated to any organisation.
\textbf{* Review team members and their organisational affiliations: Give the personal details and the organisational affiliations of each member of the review team. Affiliation refers to groups or organisations to which review team members belong.}\\
\hline
\textbf{MARS} & Not specified.\\
\hline
\end{tabular}
\end{table}

\end{collapse}

\textbf{Corresponding author}

\begin{itemize}
\tightlist
\item
  Named contact: Armin Fabian
\item
  Named contact email:
  \href{mailto:armin.fabian@uni-tuebingen.de}{\nolinkurl{armin.fabian@uni-tuebingen.de}}
\item
  Named contact ORCID:
  \href{https://orcid.org/0000-0001-6493-2147}{0000-0001-6493-2147}
\item
  Named contac address: Keplerstraße 2, room 157, 72074 Tübingen
\end{itemize}

\hypertarget{collaborators}{%
\subsection{Collaborators}\label{collaborators}}

show sources

\hypertarget{colla}{}
\begin{collapse}

\begin{table}[H]
\centering
\begin{tabular}{>{}l|l}
\hline
\cellcolor[HTML]{ececec}{source} & \cellcolor[HTML]{ececec}{description}\\
\hline
\textbf{PRISMA-P} & Not specified.\\
\hline
\textbf{PROSPERO} & Collaborators (name \& affilitation) of individuals working on the review, but are not review team member.\\
\hline
\textbf{MARS} & Not specified.\\
\hline
\end{tabular}
\end{table}

\end{collapse}

\begin{itemize}
\tightlist
\item
  Review team members and their organisational affiliations:

  \begin{itemize}
  \tightlist
  \item
    Iris Backfisch,
    \href{mailto:i.backfisch@iwm-tuebingen.de}{\nolinkurl{i.backfisch@iwm-tuebingen.de}},
    ORCID
    \href{https://orcid.org/0000-0002-1363-9888}{0000-0002-1363-9888}
  \item
    Andreas Lachner,
    \href{mailto:andreas.lachner@uni-tuebingen.de}{\nolinkurl{andreas.lachner@uni-tuebingen.de}},
    ORCID
    \href{https://orcid.org/0000-0001-5866-7164}{0000-0001-5866-7164}
  \end{itemize}
\item
  Apart from the review team members, there will be no (external)
  collaboration. \#\# Amendments to previous versions
\end{itemize}

show sources

\hypertarget{atpv}{}
\begin{collapse}

\begin{table}[H]
\centering
\begin{tabular}{>{}l|l}
\hline
\cellcolor[HTML]{ececec}{source} & \cellcolor[HTML]{ececec}{description}\\
\hline
\textbf{PRISMA-P} & If the protocol represents an amendment of a previously completed or published protocol, identify as such and list changes; otherwise, state plan for documenting important protocol amendments.\\
\hline
\textbf{PROSPERO} & Not specified.\\
\hline
\textbf{MARS} & Not specified.\\
\hline
\end{tabular}
\end{table}

\end{collapse}

Amendments will be published as new version of the document under the
same DOI (or will point to previous version).

\hypertarget{funding-sources-sponsors-and-their-roles}{%
\subsection{Funding sources, sponsors and their
roles}\label{funding-sources-sponsors-and-their-roles}}

show sources

\hypertarget{fsign}{}
\begin{collapse}

\begin{table}[H]
\centering
\begin{tabular}{>{}l|l}
\hline
\cellcolor[HTML]{ececec}{source} & \cellcolor[HTML]{ececec}{description}\\
\hline
PRISMA-P & * Indicate sources of financial or other support for the review
* Provide name for the review funder and/or sponsor
\textbf{* Describe roles of funder(s), sponsor(s), and/or institution(s), if any, in developing the protocol}\\
\hline
\textbf{PROSPERO} & Funding sources/sponsors: Give details of the individuals, organizations, groups or other legal entities who take responsibility for initiating, managing, sponsoring and/or financing the review. Include any unique identification numbers assigned to the review by the individuals or bodies listed. Grant numbers.\\
\hline
MARS & * List all sources of monetary and in-kind funding support
\textbf{* State the role of funders in conducting the synthesis and deciding to publish the results, if any}\\
\hline
\end{tabular}
\end{table}

\end{collapse}

BMBF - This project is part of the ``Qualitätsoffensive Lehrerbildung'',
a joint initiative of the Federal Government andthe Länder which aims to
improve the quality of teacher training. The programme is funded by the
Federal Ministry of Education and Research. The authors are responsible
for the content of this publication.

\hypertarget{conflict-of-interest}{%
\subsection{Conflict of Interest}\label{conflict-of-interest}}

show sources

\hypertarget{coi}{}
\begin{collapse}

\begin{table}[H]
\centering
\begin{tabular}{>{}l|l}
\hline
\cellcolor[HTML]{ececec}{source} & \cellcolor[HTML]{ececec}{description}\\
\hline
\textbf{PRISMA-P} & Not specified.\\
\hline
\textbf{PROSPERO} & List any conditions that could lead to actual or perceived undue influence on judgements concerning the main topic investigated in the review.\\
\hline
\textbf{MARS} & Describe possible conflicts of interest, including financial and other nonfinancial interests.\\
\hline
\end{tabular}
\end{table}

\end{collapse}

No conflict of interest.

\hypertarget{introduction}{%
\section{Introduction}\label{introduction}}

\hypertarget{rationale}{%
\subsection{Rationale}\label{rationale}}

show sources

\hypertarget{ram}{}
\begin{collapse}

\begin{table}[H]
\centering
\begin{tabular}{>{}l|l}
\hline
\cellcolor[HTML]{ececec}{source} & \cellcolor[HTML]{ececec}{description}\\
\hline
\textbf{PRISMA-P} & Describe the rationale for the review in the context of what is already known.\\
\hline
\textbf{PROSPERO} & Not specified.\\
\hline
MARS & Problem: State the question or relation(s) under investigation, including

* Historical background, including previous syntheses and meta-analyses related to the topic 
* Theoretical, policy, and/or practical issues related to the question or relation(s) of interest
* Populations and settings to which the question or relation(s) is relevant
* Rationale for
   (a) choice of study designs, 
   (b) the selection and coding of outcomes, 
   (c) the selection and coding potential moderators or mediators of results 
\textbf{* Psychometric characteristics of outcome measures and other variables}\\
\hline
\end{tabular}
\end{table}

\end{collapse}

When it comes to school or education as such, the discussion about the
use of digital media has become ubiquitous. However, recent studies
indicate that teachers still rarely use digital technologies for
educational purposes, and if they do, they fail to integrate them into
teaching in a didactically meaningful manner (Farjon, Smits \& Voogt,
2019). One of the main boundary conditions of successful technology
integration, that researchers have identified, is the professional
knowledge of teachers. Accordingly, to use technologies in classrooms
purposefully, teachers need specific knowledge that is tailored around
the use of digital technologies. One of the most recited and adopted
models used to describe such knowledge is the TPACK
(\textbf{t}echnological, \textbf{p}edagogical \textbf{a}nd
\textbf{c}ontent \textbf{k}nowledge) model by Mishra and Koehler (2006).
The TPACK model captures the idea of bringing together and connecting
basic knowledge components (i.e.~knowledge about technology, pedagogy
and content) to form a new central form of knowledge -- TPACK
(technological, pedagogical and content knowledge. In literature, TPACK
has evolved to become \emph{the} central focus of researchers when it
comes to knowledge regarding technology integration (Kim \& Lee, 2018).

Ever since the introduction of the TPACK model in 2006, numerous
researchers have worked \emph{on} the model trying to clarify its
underlying structure (Angeli \& Valanides, 2009; Graham, 2011); or
\emph{with} the model using it as theoretical background for data driven
studies (Angeli et al., 2016; Cavanagh \& Koehler, 2013;). Yet, to date
the question of what TPACK constitutes remains a source of scholarly
debate (Petko, 2020). The drive of this debate seems to be mainly due to
the vast, diverse and often seemingly contradictory conceptualizations
of TPACK that exist in the TPACK literature. To provide a comprehensive
picture on TPACK, it is therefore necessary to understand and organize
the different conceptualizations that researchers introduced when
working on or with the TPACK model.

Against this background, we conduct a systematic review that attempts to
clarify and systematize existing conceptualizations within the huge
corpus of TPACK research. More precisely, we are interested in examining
if TPACK researchers -- in their endeavours of conceptualizing TPACK --
have put emphasis on specific TPACK components (i.e., subdomains of
TPACK: Technological Knowledge, content knowledge, pedagogical
Knowledge, Pedagogical content knowledge, Technological content
Kowledge, technologocical pedagagogical knowledge and technological
pedagogical and content knowledge) while possibly neglecting others. In
other words, we will examine which foci lenses have been used by
researchers when dealing with the TPACK model. A particular focus will
here be on \emph{how} and to \emph{what extent} content-specific
features of TPACK were accounted for in existing conceptualizations of
the TPACK model. Moreover, we will systematically look at existing TPACK
measurement methods used in data driven studies to gain a deeper
understanding of the empirical applicability of existing TPACK
conceptualizations.

This systematic review will help to organize and understand different
existent TPACK conceptualizations in research, and thereby paves the way
for fruitful applications of this highly complex framework in the
future.

\textbf{References}

Angeli, C., \& Valanides, N. (2009). Epistemological and methodological
issues for the conceptualization, development, and assessment of
ICT-TPCK: Advances in technological pedagogical content knowledge
(TPCK). \emph{Computers \& Education}, \emph{52}(1), 154--168. doi:
10.1016/j.compedu.2008.07.006.

Angeli, C.; Voogt, J.; Fluck, A.; Webb, M.; Cox, M.; Malyn-Smith, J.,\&
Zagani, J. (2016): A K-6 Computational Thinking Curriculum Framework:
Implication for Teacher Knowledge. In: \emph{Educational Technology \&
Society}, \emph{19}(3), 47--57.

Cavanagh, Robert F.; Koehler, Matthew J. (2013): A Turn toward
Specifying Validity Criteria in the Measurement of Technological
Pedagogical Content Knowledge (TPACK). \emph{Journal of Research on
Technology in Education}, \emph{46}(2), 129-148 doi:
\url{https://doi.org/10.1080/15391523.2013.10782616}

Cox, S., \& Graham, CR (2009). Diagramming TPACK in practice: Using an
elaborated model of the TPACK framework to analyze and depict teacher
knowledge. \emph{TechTrends: Linking Research \& Practice to Improve
Learning},\emph{53}(5), 60-69. doi:
\url{https://doi.org/10.1007/s11528-009-0327-1}

Farjon, D., Smits, A., \& Voogt, J. (2019). Technology integration of
student teachers explained by attitudes and beliefs, competency, access,
and experience. \emph{Computers \& Education}, \emph{130}, 81-93. doi:
\url{https://doi.org/10.1016/j.compedu.2018.11.010}

Kim, S.-W., \& Lee, Y. J. (2018): The Effects of the TPACK-P Educational
Program on Teachers' TPACK: Programming as a Technological Tool.
\emph{International Journal of Engineering \& Technology}, \emph{30}(7),
636--643. doi: \url{https://www.doi.org/10.14419/ijet.v7i3.34.19405}

Mishra, P., \& Koehler, M. J. (2006). Technological pedagogical content
knowledge: A framework for integrating technology in teacher knowledge.
\emph{Teachers College Record}, \emph{108}(6), 1017-1054. doi:
\url{https://doi.org/10.1111/j.1467-9620.2006.00684.x}

Petko, D. (2020). Quo vadis TPACK? Scouting the road ahead.
\emph{Proceedings of EdMedia + Innovate Learning} (pp.~1349-1358).
Online, The Netherlands: Association for the Advancement of Computing in
Education (AACE). Retrieved October 10, 2020 from
\url{https://www.learntechlib.org/primary/p/217445/}.

\hypertarget{research-questions}{%
\subsection{Research Questions}\label{research-questions}}

show sources

\hypertarget{rq}{}
\begin{collapse}

\begin{table}[H]
\centering
\begin{tabular}{>{}l|l}
\hline
\cellcolor[HTML]{ececec}{source} & \cellcolor[HTML]{ececec}{description}\\
\hline
\textbf{PRISMA-P} & Provide an explicit statement of the question(s) the review will address with reference to participants, interventions, comparators, and outcomes (PICO)\\
\hline
\textbf{PROSPERO} & State the question(s) to be addressed by the review, clearly and precisely. Review questions may be specific or broad. It may be appropriate to break very broad questions down into a series of related more specific questions. Questions may be framed or refined using PI(E)COS where relevant.\\
\hline
MARS & Objectives: State the hypotheses examined, indicating which were prespecified, including

* Question in terms of relevant participant characteristics (including animal populations), independent variables (experimental manipulations, treatments, or interventions), ruling out of possible confounding variables, dependent variables (outcomes, criterion), and other features of study designs
\textbf{* Method(s) of synthesis and if meta-analysis was used, the specific methods used to integrate studies (e.g., effect-size metric, averaging method, the model used in homogeneity analysis)}\\
\hline
\end{tabular}
\end{table}

\end{collapse}

\textbf{RQ1)} How is teachers' technological-pedagogical content
knowledge (TPACK) conceptualized within research? What different
components are defined and on which component does the focus lie (T, P,
C, TP, TC, TP, PC, TPC)?

\textbf{RQ2a)} What kind of test instruments have been used to measure
TPACK in empirical studies and in which ways to the differ in their
application?

\textbf{RQ2b)} How do the several existing TPACK measurement instruments
mirror the distinct conceptualizations of TPACK?

\hypertarget{methods}{%
\section{Methods}\label{methods}}

\hypertarget{eligibility-inclusion-and-exclusion-criteria}{%
\subsection{Eligibility: Inclusion and Exclusion
Criteria}\label{eligibility-inclusion-and-exclusion-criteria}}

show sources

\hypertarget{eiaec}{}
\begin{collapse}

\begin{table}[H]
\centering
\begin{tabular}{>{}l|l}
\hline
\cellcolor[HTML]{ececec}{source} & \cellcolor[HTML]{ececec}{description}\\
\hline
\textbf{PRISMA-P} & Specify the study characteristics (such as PICO, study design, setting, time frame) and report characteristics (such as years considered, language, publication status) to be used as criteria for eligibility for the review\\
\hline
\textbf{PROSPERO} & Give details of the types of study (study designs) eligible for inclusion in the review. If there are no restrictions on the types of study design eligible for inclusion, or certain study types are excluded, this should be stated. The preferred format includes details of both inclusion and exclusion criteria.\\
\hline
MARS & Describe the criteria for selecting studies, including

* Independent variables (e.g., experimental manipulations, types of treatments or interventions or predictor variables)
* Dependent variable (e.g., outcomes, in syntheses of clinical research including both potential benefits and potential adverse effects)
* Eligible study designs (e.g., methods of sampling or treatment assignment)
* Handling of multiple reports about the same study or sample, describing which are primary and handling of multiple measures using the same participants
* Restrictions on study inclusion (e.g., by study age, language, location, or report type)
* Changes to the prespecified inclusion and exclusion criteria, and when these changes were made
\textbf{* Handling of reports that did not contain sufficient information to judge eligibility (e.g., lacking information about study design) and reports that did not include sufficient information for analysis (e.g., did not report numerical data about those outcomes)}\\
\hline
\end{tabular}
\end{table}

\end{collapse}

\hypertarget{inclusion-criteria}{%
\subsubsection{Inclusion criteria}\label{inclusion-criteria}}

\begin{itemize}
\item
  Papers that investigate teachers' professional knowledge for
  technology-enhanced teaching based on the TPACK framework by Mishra \&
  Koehler (2006) or adoptions and extensions thereof
\item
  Theoretical contributions that conceptualize TPACK (or adoptions and
  extensions thereof)
\item
  Empirical studies that assess TPACK of instructiors such as
  (pre-service/in-service) teachers across all levels, university
  professors, tutors etc.
\item
  Peer-reviewed journal articles, conference proceedings and
  dissertations
\end{itemize}

\hypertarget{exclusion-criteria}{%
\subsubsection{Exclusion criteria}\label{exclusion-criteria}}

\begin{itemize}
\item
  No fulltext available
\item
  Papers that are not written in English
\item
  Papers in which TPACK (or adoptions and extensions thereof) was not
  explicitly mentioned in the title or abstract
\item
  Papers from which no clear conceptualization of TPACK can be made out
  (in the sense of RQ 2, see 2.2)
\end{itemize}

\hypertarget{sources-of-search-list-and-rationale}{%
\subsection{Sources of Search: List and
Rationale}\label{sources-of-search-list-and-rationale}}

show sources

\hypertarget{soslar}{}
\begin{collapse}

\begin{table}[H]
\centering
\begin{tabular}{>{}l|l}
\hline
\cellcolor[HTML]{ececec}{source} & \cellcolor[HTML]{ececec}{description}\\
\hline
\textbf{PRISMA-P} & Not specified.\\
\hline
PROSPERO & Searches: State the sources that will be searched. Give the search dates, and any restrictions (e.g. language or
\textbf{publication period). Do NOT enter the full search strategy (it may be provided as a link or attachment.)}\\
\hline
MARS & Describe all information sources:

* Databases searched (e.g., PsycINFO, ClinicalTrials.gov), including dates of coverage (i.e., earliest and latest records included in the search), and software and search platforms used
* Names of specific journals that were searched and the volumes checked
* Explanation of rationale for choosing reference lists if examined (e.g., other relevant articles, previous research
syntheses)
* Documents for which forward (citation) searches were conducted, stating why these documents were chosen
* Number of researchers contacted if study authors or individual researchers were contacted to find studies or to obtain more information about included studies, as well as criteria for making contact (e.g., previous relevant publications), and response rate
* Dates of contact if other direct contact searches were conducted such as contacting corporate sponsors or mailings to distribution lists
\textbf{* Search strategies in addition to those above and the results of these searches}\\
\hline
\end{tabular}
\end{table}

\end{collapse}

\begin{itemize}
\tightlist
\item
  data bases:

  \begin{itemize}
  \tightlist
  \item
    Web of Science
  \item
    PsychINFO
  \item
    ERiC
  \item
    ScienceDirect
  \item
    ProQuest Dissertations
  \item
    first 100 results from google scholar
  \end{itemize}
\item
  backwards search

  \begin{itemize}
  \tightlist
  \item
    After screening abstracts and titles: We use the three latest
    reviews on TPACK (or adoptions thereof) and screen their references
  \end{itemize}
\end{itemize}

\hypertarget{search-strategy}{%
\subsection{Search Strategy}\label{search-strategy}}

show sources

\hypertarget{searchs}{}
\begin{collapse}

\begin{table}[H]
\centering
\begin{tabular}{>{}l|l}
\hline
\cellcolor[HTML]{ececec}{source} & \cellcolor[HTML]{ececec}{description}\\
\hline
\textbf{PRISMA-P} & Present draft of search strategy to be used for at least one electronic database, including planned limits, such that it could be repeated.\\
\hline
\textbf{PROSPERO} & URL to search strategy: Give a link to a published pdf/word document detailing either the search strategy or an example of a search strategy for a specific database if available (including the keywords that will be used in the search strategies), or upload your search strategy. Do NOT provide links to your search results. Alternatively, upload your search strategy to CRD in pdf format. Please note that by doing so you are consenting to the file being made publicly accessible.\\
\hline
\textbf{MARS} & Describe all information sources: Search strategies of electronic searches, such that they could be repeated (e.g., include the search terms used, Boolean connectors, fields searched, explosion of terms).\\
\hline
\end{tabular}
\end{table}

\end{collapse}

\textbf{Search String}

((TPACK OR TPACK OR ``technological pedagogical content knowledge'' OR
``technological-pedagogical-content-knowledge'' OR ``technological
pedagogical and content knowledge'') AND teacher*)

\textbf{Additional Specifications Used}

\begin{itemize}
\tightlist
\item
  language: English
\item
  time span: 2005-2020
\end{itemize}

\textbf{Note}: In 2005, the acrononym TPACK was first used by Mishra and
Koehler. Thus, to offer a complete picture on existing TPACK
conceptualizations, all TPACK contributions that have been published
since 2005 are taken into consideration.

\hypertarget{data-management-tools-used}{%
\subsection{Data Management Tools
Used}\label{data-management-tools-used}}

show sources

\hypertarget{dmtu}{}
\begin{collapse}

\begin{table}[H]
\centering
\begin{tabular}{>{}l|l}
\hline
\cellcolor[HTML]{ececec}{source} & \cellcolor[HTML]{ececec}{description}\\
\hline
\textbf{PRISMA-P} & Describe the mechanism(s) that will be used to manage records and data throughout the review.\\
\hline
\textbf{PROSPERO} & Not specified.\\
\hline
\textbf{MARS} & Not specified.\\
\hline
\end{tabular}
\end{table}

\end{collapse}

\begin{itemize}
\tightlist
\item
  Rayyan (\url{https://rayyan.qcri.org/})
\item
  Citavi (\url{https://www.citavi.com/de})
\end{itemize}

\hypertarget{data-extraction-selection-of-studies}{%
\subsection{Data Extraction (Selection of
Studies)}\label{data-extraction-selection-of-studies}}

show sources

\hypertarget{desos}{}
\begin{collapse}

\begin{table}[H]
\centering
\begin{tabular}{>{}l|l}
\hline
\cellcolor[HTML]{ececec}{source} & \cellcolor[HTML]{ececec}{description}\\
\hline
\textbf{PRISMA-P} & State the process that will be used for selecting studies (such as two independent reviewers) through each phase of the review (that is, screening, eligibility and inclusion in meta-analysis).\\
\hline
\textbf{PROSPERO} & Data extraction (selection and coding): Describe how studies will be selected for inclusion. State what data will be extracted or obtained. State how this will be done and recorded.\\
\hline
MARS & Describe the process for deciding which studies would be included in the syntheses and/or included in the meta-analysis, including

* Document elements (e.g., title, abstract, full text) used to make decisions about inclusion or exclusion from the synthesis at each step of the screening process 
\textbf{* Qualifications (e.g., training, educational or professional status) of those who conducted each step in the study selection process, stating whether each step was conducted by a single person or in duplicate as well as an explanation of how reliability was assessed if one screener was used and how disagreements were resolved if multiple were used.}\\
\hline
\end{tabular}
\end{table}

\end{collapse}

Two independent reviewers conduct every of the following steps:

\begin{enumerate}
\def\labelenumi{\arabic{enumi})}
\tightlist
\item
  screening titles and if we cannot make a statement based on the title
  we will screen the abstract
\item
  screening all abstracts
\item
  screening full texts
\end{enumerate}

At each of this three steps, articles will be included if the inclusion
criteria apply and none of the exclusivity criteria apply. If this is
not the case, articles will be excluded. If it is not clear whether
articles should be included or excluded, these articles will be labelled
as ``maybe'' and then discussed among the raters until consens is
reached. If there are disagreements aomng the raters regarding the
inclusion or exclusion of a certain publication, this publication will
be discussed together in more detail until consens is reached.

\hypertarget{method-of-extracting-data-information-from-reports}{%
\subsection{Method of Extracting Data \& Information (from
Reports)}\label{method-of-extracting-data-information-from-reports}}

show sources

\hypertarget{edfr}{}
\begin{collapse}

\begin{table}[H]
\centering
\begin{tabular}{>{}l|l}
\hline
\cellcolor[HTML]{ececec}{source} & \cellcolor[HTML]{ececec}{description}\\
\hline
\textbf{PRISMA-P} & Describe planned method of extracting data from reports (such as piloting forms, done independently, in duplicate), any processes for obtaining and confirming data from investigators.\\
\hline
\textbf{PROSPERO} & Not specified.\\
\hline
MARS & Describe methods of extracting data from reports, including 

* Variables for which data were sought and the variable categories 
\textbf{* Qualifications of those who conducted each step in the data extraction process, stating whether each step was conducted by a single person or in duplicate and an explanation of how reliability was assessed if one screener was used and how disagreements were resolved if multiple screeners were used as well as whether data coding forms, instructions for completion, and the data (including metadata) are available, stating where they can be found (e.g., public registry, supplemental materials)}\\
\hline
\end{tabular}
\end{table}

\end{collapse}

\begin{itemize}
\item
  Studies will be coded in Rayyan by two independent raters using the
  ``inclusion'', ``exclusion'' and ``maybe'' labelling function.
\item
  To extract detailed information on the included studies, a
  standardized Excel or self-programmed dashboard that produces a
  relational database will be established and applied by two independent
  raters.
\item
  All extracted data and information will be analyzed for inter-rater
  agreement and discrepancies will be discussed (see also 3.5).
\end{itemize}

\hypertarget{list-and-description-of-data-and-information-extracted}{%
\subsection{List and Description of Data and Information
Extracted}\label{list-and-description-of-data-and-information-extracted}}

show sources

\hypertarget{ladodaie}{}
\begin{collapse}

\begin{table}[H]
\centering
\begin{tabular}{>{}l|l}
\hline
\cellcolor[HTML]{ececec}{source} & \cellcolor[HTML]{ececec}{description}\\
\hline
PRISMA-P & * List and define all variables for which data will be sought (such as PICO items, funding sources), any pre-planned data assumptions and simplifications
\textbf{* List and define all outcomes for which data will be sought, including prioritization of main and additional outcomes, with rationale}\\
\hline
PROSPERO & * Condition or domain being studied: Give a short description of the disease, condition or healthcare domain being studied. This could include health and wellbeing outcomes.
* Participants/population: Give summary criteria for the participants or populations being studied by the review. The preferred format includes details of both inclusion and exclusion criteria.
* Intervention(s), exposure(s): Give full and clear descriptions or definitions of the nature of the interventions or the exposures to be reviewed.
* Comparator(s)/control: Where relevant, give details of the alternatives against which the main subject/topic of the review will be compared (e.g. another intervention or a non-exposed control group). The preferred format includes details of both inclusion and exclusion criteria.
* Main and additional outcome(s): Give the pre-specified main (most important) outcomes of the review, including details of how the outcome is defined and measured and when these measurement are made, if these are part of the review inclusion criteria.
* Measures of effect: Please specify the effect measure(s) for you main outcome(s) e.g. relative risks, odds ratios, risk difference,
\textbf{and/or 'number needed to treat.}\\
\hline
\textbf{MARS} & Not specified.\\
\hline
\end{tabular}
\end{table}

\end{collapse}

\begin{itemize}
\tightlist
\item
  \emph{General information on the publication}, such as publication
  status, publication year, authors' names, type of paper (theoretical,
  empirical, etc)
\item
  \emph{Specific information on the authors}, such as discipline of the
  authors (general teacher educators, content specialists, education
  professors, etc.)
\item
  \emph{In the case of empirical studies} : sample size, sample group
  (students, pre-service teachers, in-service teachers K-12, etc.),
  country the study was conducted, gender, subject domain, which kind of
  TPACK measurement was used (self-report, performance-based,
  observations, etc.)
\end{itemize}

\emph{Note}: This list might be subject to extensions and/or adoptions
as we start with the revision process.

\hypertarget{effect-size-transformation-from-individual-studies}{%
\subsection{Effect size transformation from individual
studies}\label{effect-size-transformation-from-individual-studies}}

show sources

\hypertarget{estfis}{}
\begin{collapse}

\begin{table}[H]
\centering
\begin{tabular}{>{}l|l}
\hline
\cellcolor[HTML]{ececec}{source} & \cellcolor[HTML]{ececec}{description}\\
\hline
\textbf{PRISMA-P} & Not specified.\\
\hline
\textbf{PROSPERO} & Not specified.\\
\hline
MARS & Describe the statistical methods for calculating effect sizes, including the metric(s) used (e.g., correlation coefficients,
\textbf{differences in means, risk ratios) and formula(s) used to calculate effect sizes.}\\
\hline
\end{tabular}
\end{table}

\end{collapse}

not relevant as this is a review, not a meta-analysis.

\hypertarget{risk-of-bias-in-individual-studies}{%
\subsection{Risk of Bias in Individual
Studies}\label{risk-of-bias-in-individual-studies}}

show sources

\hypertarget{robiis}{}
\begin{collapse}

\begin{table}[H]
\centering
\begin{tabular}{>{}l|l}
\hline
\cellcolor[HTML]{ececec}{source} & \cellcolor[HTML]{ececec}{description}\\
\hline
\textbf{PRISMA-P} & Describe anticipated methods for assessing risk of bias of individual studies, including whether this will be done at the outcome or study level, or both; state how this information will be used in data synthesis\\
\hline
\textbf{PROSPERO} & Risk of bias (quality) assessment: Describe the method of assessing risk of bias or quality assessment. State which characteristics of the studies will be assessed and any formal risk of bias tools that will be used.\\
\hline
MARS & Describe any methods used to assess risk to internal validity in individual study results, including

* Risks assessed and criteria for concluding risk exists or does not exist
\textbf{* Methods for including risk to internal validity in the decisions to synthesize of the data and the interpretation of results}\\
\hline
\textbf{Added by authors} & Describe how the quality of original studies are rated. E.g. by 'The Study Design and Implementation Assessment Device (Study DIAD)': https://doi.org/10.1037/1082-989X.13.2.130\\
\hline
\end{tabular}
\end{table}

\end{collapse}

Through the systematic approach used to obtain our final sample, the
general quality of included publications should be high. To account for
individual difficulties and possible bias within publications, a
qualitative content analysis approach will be conducted for each
publication individually. By doing so, we can make sure to detect and
properly reflect upon (empirical) difficulties that we come across in
individual publications.

\hypertarget{results}{%
\section{Results}\label{results}}

\hypertarget{strategy-for-data-synthesis}{%
\subsection{Strategy for Data
Synthesis}\label{strategy-for-data-synthesis}}

show sources

\hypertarget{sfds}{}
\begin{collapse}

\begin{table}[H]
\centering
\begin{tabular}{>{}l|l}
\hline
\cellcolor[HTML]{ececec}{source} & \cellcolor[HTML]{ececec}{description}\\
\hline
PRISMA-P & * Describe criteria under which study data will be quantitatively synthesised.
* If data are appropriate for quantitative synthesis, describe planned summary measures, methods of handling data and methods of combining data from studies, including any planned exploration of consistency (such as I2, Kendall’s t)
\textbf{* If quantitative synthesis is not appropriate, describe the type of summary planned}\\
\hline
\textbf{PROSPERO} & Strategy for data synthesis: Provide details of the planned synthesis including a rationale for the methods selected. This must not be generic text but should be specific to your review and describe how the proposed analysis will be applied to your data.\\
\hline
MARS & Describe narrative and statistical methods used to compare studies. If meta-analysis was conducted, describe the methods used to combine effects across studies and the model used to estimate the heterogeneity of the effects sizes(e.g., a fixed-effect, random-effects model robust variance estimation), including

* Rationale for the method of synthesis
* Methods for weighting study results
* Methods to estimate imprecision (e.g., confidence or credibility intervals) both within and between studies
* Description of all transformations or corrections (e.g., to account for small samples or unequal group numbers) and adjustments (e.g., for clustering, missing data, measurement artifacts, or construct-level relationships) made to the data and justification for these
* Additional analyses (e.g., subgroup analyses, meta-regression), including whether each analysis was prespecified or post hoc
* Selection of prior distributions and assessment of model fit if Bayesian analyses were conducted
* Name and version number of computer programs used for the analysis
\textbf{* Statistical code and where it can be found (e.g., a supplement)}\\
\hline
\end{tabular}
\end{table}

\end{collapse}

After the final selection of the sample from the revision process, a
content analysis approach will be conducted (see Mayring, 2014). This
means, that for each of the selected publication, relevant information
will be clustered and organized into units of meaning. These units of
meaning carry information that will help in answering the research
questions.

Our analytical approach can be considered both deductive as well as
inductive. It will be deductive in the sense that our starting point for
developing the coding scheme will be the TPACK model and the complex
interplay of its subdomains. On the other hand, our approach includes
inductive characteristics as our coding scheme might be supplemented by
further labels extracted from individual contributions as we conduct our
sample.

Our dichotomous approach will contribute to a comprehensive
understanding of TPACK helping future researchers to theoratically and
empirically apply the complex TPACK framework.

\end{document}
